\documentclass{article}
% Change "article" to "report" to get rid of page number on title page
\usepackage{amsmath,amsfonts,amsthm,amssymb}
\usepackage{setspace}
\usepackage{Tabbing}
\usepackage{fancyhdr}
\usepackage{lastpage}
\usepackage{extramarks}
\usepackage{url}
\usepackage{chngpage}
\usepackage{longtable}
%\usepackage{subfigure}
\usepackage{soul,color}
\usepackage{graphicx,float,wrapfig}
%\usepackage{caption,subcaption}
\usepackage{enumitem}
\usepackage{morefloats}
\usepackage{multirow}
\usepackage{multicol}
\usepackage{indentfirst}
\usepackage{lscape}
\usepackage{pdflscape}
\usepackage{natbib}
\usepackage[toc,page]{appendix}
\providecommand{\e}[1]{\ensuremath{\times 10^{#1} \times}}

% In case you need to adjust margins:
%\topmargin=-0.45in      % Switch to the other top for overleaf
\topmargin=0.25in      %
\evensidemargin=0in     %
\oddsidemargin=0in      %
\textwidth=6.5in        %
%\textheight=9.75in       % play with this for overleaf
\textheight=9.25in       %
\headsep=0.25in         %

% Homework Specific Information
\newcommand{\hmwkTitle}{Introductions}
\newcommand{\hmwkDueDate}{due 12:00 Monday,\ August\  27,\ 2018}
\newcommand{\hmwkClass}{Homework 0}
\newcommand{\hmwkClassTime}{CSE 597.2 -- Implementation of Parallel Programming Codes}
\newcommand{\hmwkClassInstructor}{ } \newcommand{\hmwkAuthorNameb}{Zachary Moon} %{XYZ\ ZYX.\ YZX}
\newcommand{\hmwkNames}{Zachary Moon (zlm1)}

% Setup the header and footer
\pagestyle{fancy}
\lhead{\hmwkNames}
\rhead{\hmwkClass: \hmwkTitle} 
\cfoot{Page\ \thepage\ of\ \pageref{LastPage}}
\renewcommand\headrulewidth{0.4pt}
\renewcommand\footrulewidth{0.4pt}




%%%%%%%%%%%%%%%%%%%%%%%%%%%%%%%%%%%%%%%%%%%%%%%%%%%%%%%%%%%%%
% Make title
\title{\vspace{2in}\textmd{\textbf{\hmwkClass:\ \hmwkTitle}}\\\normalsize\vspace{0.1in}\small{\hmwkDueDate}\\\vspace{0.1in}\large{\textit{\hmwkClassInstructor\ \hmwkClassTime}}\vspace{3in}}
\date{}
\author{\textbf{\hmwkAuthorNameb} } % \\ \textbf{\hmwkAuthorNamea}}
%%%%%%%%%%%%%%%%%%%%%%%%%%%%%%%%%%%%%%%%%%%%%%%%%%%%%%%%%%%%%

\begin{document}
\begin{spacing}{1.1}
\maketitle

\newpage
\section{Syllabus Acknowledgement}

By turning in this assignment, I, Zachary Moon, acknowledge that I have received and understand the course syllabus information available on \url{sites.psu.edu/psucse597fall2018}. 

\section{Introduction}

My name is Zachary Moon.  I am a 3rd year PhD student in the Meteorology \& Atmospheric Science department. My programming experience includes some numerical modeling with Fortran (with some OpenMP experience) and Python, and data analysis with Python, Matlab, R, NCL, etc.  When I compute, I typically use ICS ACI (aci-b) or Linux clusters in my department.  My research so far has been mostly computational in nature. 

My area of interest is modeling photochemistry within plant canopies. Good general references in my field are \citet{fuentes_biogenic_2000}, \citet{dickinson_land_1983}, and \citet{myneni_review_1989}. I'm not really sure of good computational references in my field. 


\subsection{Accounts}

I have gotten an account on ACI using \url{https://ics.psu.edu/?page_id=57}. My ACI username is zlm1.

I have gotten an account on XSEDE using `create account' on \url{https://portal.xsede.org}. 
My username is zlm1.

I will be making my assignments available using GitHub and ACI git repositories. My GitHub username is zmoon92. 

\subsection{My Course Project}

I am currently thinking about choosing satellite retrieval of vegetation features such as leaf area index (LAI) as my $Ax=b$ problem for the semester project. I believe that this will be a good project because
\begin{itemize}
  \item I have been interested in learning more about this topic for a while
  \item I can use my canopy radiative transfer codes I have been developing for part of the inverse problem
\end{itemize}


\section{HW 0 Code and Writeup}

You can get my assignment onto ACI using the command:
\begin{verbatim}
git clone USERID@aci-b.aci.ics.psu.edu:/storage/home/z/zlm1/work/cse-597-2/remotes/hw0
\end{verbatim}
or if already on ACI, simply
\begin{verbatim}
git clone /storage/home/z/zlm1/work/cse-597-2/remotes/hw0
\end{verbatim}
or by cloning my class GitHub repository, which can be found here: \url{https://github.com/zmoon92/PSU-CSE-597.2}. 


\subsection{Program overview}

This is a serial hello world program, written in Fortran. There is only one code file. The repository also contains the makefile for creating the executable, a readme, licensing information and the tex file for the write-up.


\subsection{Instructions for running and verifying the code}

\textbf{Creating the executable:}
\begin{verbatim}
module load gcc
make
\end{verbatim}

\textbf{Running the program:}
\begin{verbatim}
./hi.out
\end{verbatim}

\textbf{Expected output:}
\begin{verbatim}
"Hello, World!"
\end{verbatim}

\subsection{Instructions for compiling the write-up}

I used ACI to compile the document. You can do this using the command:
\begin{verbatim}
./pdfmake.sh
\end{verbatim}

\section{Acknowledgements}

I would like to acknowledge Chris Blanton and Chuck Pavloski for helping formulate the homework material, and Justin Petucci and Rahim Charania for helping to make sure the permissions were set correctly for the git information for the template, and Adam Lavely for making the template. 

\bibliographystyle{agu}
\bibliography{hw0_writeup_refs_zlm1}

\end{spacing}

\end{document}

%%%%%%%%%%%%%%%%%%%%%%%%%%%%%%%%%%%%%%%%%%%%%%%%%%%%%%%%%%%%%}}
